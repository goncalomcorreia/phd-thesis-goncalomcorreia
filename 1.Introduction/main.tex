% %%%%%%%%%%%%%%%%%%%%%%%%%%%%%%%%%%%%%%%%%%%%%%%%%%%%%%%%%%%%%%%%%%%%%%
% The Introduction:
% %%%%%%%%%%%%%%%%%%%%%%%%%%%%%%%%%%%%%%%%%%%%%%%%%%%%%%%%%%%%%%%%%%%%%%
\fancychapter{Introduction}
\label{cap:int}

The success of well known deep learning
models~\citep{convnet,devlin2018bert,brown2020language} rely on a
rich parameterization through the use of numerous neural network
layers. Along with huge amounts of datapoints, this allows for these
models to learn good vector representations that consequently permit
these models to excel in their respective tasks. However, these
models are often highly overparameterized and their interpretation is
difficult. Moreover, they require very expensive computing power,
which causes understandable environmental
concerns~\citep{Strubell2019}. In natural language processing (NLP),
one such model is the Transformer
architecture~\citep{vaswani2017attention} which has quickly risen to
prominence through its performance, leading to improvements in the
state of the art of neural machine
translation~\citep[NMT;][]{marian,ott2018scaling}, and served as
inspiration to even bigger and powerful general-purpose models like
BERT~\citep{devlin2018bert} and
\mbox{GPT-3}~\citep{brown2020language}.

On the other hand, neural latent variable models are powerful and
expressive tools for finding patterns in high-dimensional data, such
as images or text~\citep{Kim2018,Kingma+2014:VAE,RezendeEtAl14VAE}.
These models have powerful structural biases that guide the model's
training, leading to models that are more transparent. Of particular
interest are \emph{discrete} latent variables, which can recover
categorical and structured encodings of hidden aspects of the data,
leading to compact representations~\citep{KingmaEtAl2014SSVAE} and,
in some cases, superior explanatory power~\citep{titov2008joint,
    Bastings2019}.

Latent variables in machine translation were widely used before the
success of deep learning models and consequently NMT. Typically, the
latent variables in these models were of structured nature, such as
representing alignments of phrases in the source and target
sentence~\citep{brown-etal-1993-mathematics}. These structured
models ended up being abandoned in NMT in favour of a direct
decomposition of the model distribution via chain rule without making
any Markov assumptions, which was enabled by advances in
architectures and optimization procedures.

Recently, variational
methods have paved the way for latent variables to be used in NMT,
although usually using a continuous latent variable to represent the
semantic space of the
translation~\citep{zhang2016VariationalNeuralMachine}.
We propose to incorporate structured latent variables into NMT. We
focus on the use of these latent variables as a way to incrementally
guide the model into decomposing translation as a sequence of
subtasks, \eg first encountering a suitable draft translation from
memory and then editing that draft to create the final target
translated sentence, rather than translating from source to target in
one step. Thanks to a probabilistic framework, our methods allow
monolingual data from the source and/or target language, such that it
can be used as semi-supervision of the final NMT model. Our contributions
allow for future NMT models to have:

\begin{itemize}
    \item {\bf Inductive bias}: incorporated into the computational
          graph through the structured latent variables;
    \item {\bf Data efficiency}: thanks to the decomposition of a
          hard task into easier sub-tasks;
    \item {\bf Interpretability}: due to the explanatory power of
          latent variables with a discrete structure.
\end{itemize}

\section{Motivation, Objectives, and Scope}
\label{sec:int_motivation}



\section{Contributions and Thesis Statement}
\label{sec:int_contributions}

We will now summarize the contributions of this thesis, which will
address the open questions left to answer in the previous section.

\begin{itemize}

      \item \textbf{We use weak supervision by leveraging transfer learning
                  for data efficient sequence-to-sequence models.}
            We show how to leverage a pre-trained large Transformer
            encoder to perform a sequence-to-sequence task on a very
            small dataset. We explore different avenues of parameter
            sharing and initialization in order to make this
            possible.


      \item \textbf{We propose {\boldmath $\alpha$}-\entmaxtext{} with
                  learnable {\boldmath $\alpha$}, a new sparse probability
                  normalization function that learns its own sparsity.}
            We derive the gradient of $\alpha$ in
            $\alpha$-\entmaxtext{}~\citep{entmax}. By letting a
            Transformer learn through this gradient the $\alpha$ of
            each attention head, the attention heads can dynamically
            change their own sparsity during training. This way, each
            attention head can accommodate its sparsity to the role
            it will play in the overall model.

      \item \textbf{We conduct an extensive analysis on the increased
                  transparency of Transformer models that use {\boldmath
                              $\alpha$}-\entmaxtext{} as its normalization function in the
                  attention mechanism.}
            Besides studying the distribution of sparsity and respective $\alpha$ values
            throughout all attention heads in the Transformer, we also identify
            examples of sharper attention head behavior than what was found in
            previous work, along with the disentanglement of new found behaviors
            thanks to our proposed sparsity.

      \item \textbf{We propose a new method to train discrete and
                  structured latent variable models, based on the
                  efficiency of marginalizing expectations using
                  sparsity.}
            Thanks to this method, we can train these families of
            latent variable models without recurring to any Monte Carlo
            estimation or relaxations of the discreteness into the
            continuous space. In particular, in the unstructured
            case, our method relies on the sparsemax activation
            function. In the structured case, we propose to either
            use SparseMAP~\citep{niculae2018sparsemap} or the
            novel top-$k$ sparsemax.

      \item \textbf{We provide open-source code for each of the
                  methods we have proposed.}
            The respective repositories can be found in each
            of the chapters.


\end{itemize}

\paragraph{Thesis Statement.} The primary claim of this thesis is
that, unlike what is insisted in numerous previous work, neural
models have the capability of being data-efficient, transparent, and
compact: one only needs to look through a different lens that is
capable of leveraging weak supervision, sparsity, and latent
representations. We come to the conclusion that a \textit{vanilla}
application of neural models to the problem at hand is not sufficient
to have these properties: for \textbf{data-efficiency}, we find that we need
to allow parameter sharing, careful initialization, and powerful
transfer learning capabilities in order to succeed at a low-resource
generation task; for \textbf{transparency}, we can allow the model to have
sparsity and learn it according to its needs at training time; and
for \textbf{compactness}, we can leverage discrete latent variable models in a
better way than what was previously possible by using an exact but
efficient gradient.

\section{Publications}
\label{sec:int_publications}

During this PhD, I have co-authored the following work,
most of which will be shown within this proposal:

\begin{itemize}

    \item {\bf A Simple and Effective Approach to Automatic
          Post-Editing with Transfer Learning}~\citep{Correia2019}.
          Described in Chapter \ref{chap:ape}, this paper was accepted as a
          poster at ACL 2019.
          
    \item {\bf Unbabel's Submission to the WMT2019 APE Shared Task:
          BERT-based Encoder-Decoder for Automatic
          Post-Editing}~\citep{lopes2019unbabels}. Not included within this
          proposal, this work describes an Automatic Post-Editing model
          that was submitted to the APE Shared Task at WMT2019. This model
          won the Shared Task, obtaining state-of-the-art results in APE at
          the time.
          
    \item {\bf Adaptively Sparse
          Transformers}~\citep{correia2019adaptively}. Described in Chapter
          \ref{chap:adaptsparse}, this paper was accepted for an oral
          presentation at EMNLP 2019.
          
    \item {\bf Efficient Marginalization of Discrete and Structured
          Latent Variables via Sparsity}~\citep{correia2020procneurips}.
          Described in Chapter \ref{chap:sparsemarg}, this work was
          accepted as a spotlight paper at NeurIPS 2020.
          
\end{itemize}
\section{Roadmap}
\label{sec:int_roadmap}

Herein we show the outline of the remainder of this thesis.

We begin in \chapref{chap:background} by reviewing major concepts
that are essential to understand the content of this thesis,
particularly neural network models for NLP, sparse probability
normalization functions, and latent variable models.

In \chapref{chap:ape}, we develop a sequence-to-sequence model for
Automatic Post-Editing using the Transformer architecture and by
harnessing the transfer learning power of
BERT~\citep{devlin2018bert}.

In \chapref{chap:adaptsparse}, we augment the Transformer to use
attention modules that allow sparse probabilities that adaptively
change their own sparsity depending on their role within the model.

In \chapref{chap:sparsemarg}, we propose a new training method of
discrete and structured neural latent variable models, which uses
sparse probabilities in order to compute the training objective of
these models, instead of using Monte Carlo methods.

Finally, in \chapref{chap:conclusions}, we summarize the
contributions of the present thesis, address some of the limitations
and open problems of the present work, and discuss exciting future
directions of further research.


\cleardoublepage
